% CV universitaire de Manuel Pégourié-Gonnard.
%
% Écrit en 2008, 2009 par Manuel Pégourié-Gonnard.
% Ce fichier N'est PAS sous une licence libre, pour des raisons évidentes.
%
% Cependant, vous êtes parfaitement libres de ré-utiliser, diffuser et adapter
% à votre goût le contenu du préambule.

\documentclass[12pt, a4paper]{article}

\usepackage{xltxtra}
\usepackage{geometry}
\usepackage{xspace}
\usepackage[frenchb]{babel}
\usepackage[bookmarks=false, colorlinks=true, urlcolor=black]{hyperref}
\frenchbsetup{og=«,fg=»}

\defaultfontfeatures{Scale=MatchLowercase, Mapping=tex-text}
\setmainfont[Numbers=OldStyle]{Minion Pro}
\setsansfont[Numbers=OldStyle, BoldFont={%
  Myriad Pro Semibold}, BoldItalicFont={%
  Myriad Pro Semibold Italic}]{Myriad Pro}

\renewcommand\thepage{\arabic{page}/2}

\newcommand\upmc{université Pierre et Marie Curie\xspace}
\newcommand\email[1]{%
  \href{mailto:#1}{#1}}
\newcommand\rmurl[1]{%
  \href{#1}{\textrm{#1}}}

\newlength\dlmargin \setlength\dlmargin{6em}
\newenvironment{dateliste}{%
  \setlength\leftmargini{\dlmargin}%
  \setlength\labelsep{0pt}%
  \renewcommand\descriptionlabel[1]{%
    \makebox[\dlmargin][l]{\sffamily##1}}%
  \begin{description}
    }{%
  \end{description}}
\newcommand*\annee[1]{%
  \item[#1]}
\newcommand*\contexte[1]{%
  \par\medskip\noindent
  \textit{#1.}}

\newcommand*\partie[1]{%
  \begin{center}
  \large\sffamily\bfseries
	\hrule
    \strut#1\\
	\hrule
  \end{center}
  \nobreak}

\begin{document}

\noindent
\begin{minipage}[t]{0.6\linewidth}\obeylines
  Manuel \bsc{Pégourié-Gonnard}
  24 rue Geoffroy Saint-Hilaire
  75005 Paris
  \rmurl{http\string://people.math.jussieu.fr/\string~mpg/}
  \email{mpg@math.jussieu.fr}
\end{minipage}%
\begin{minipage}[t]{0.4\linewidth}\raggedleft\obeylines
  Né le 2 mars 1981
  Nationalité française
\end{minipage}
\par
\vspace{1.2em}

\partie{Formation scientifique}
\begin{dateliste}
  \annee{2005--}
  Doctorant de l'université Pierre et Marie Curie, institut de
  mathématiques de Jussieu. Directeur : Patrice Philippon. Sujet :
  Problèmes quantitatifs en approximation diophantienne dans les variétés
  abéliennes.
  \annee{2005} Reçu 58\ieme à l'agrégation de mathématiques.
  \annee{2003--2005} DEA/M2 « algèbre et géométrie» de l'\upmc,\\ mention très
  bien.
  \annee{2002--2003} Maîtrise de mathématiques de l'\upmc.
  \annee{2000--2001} Licence de mathématiques de l'\upmc.
  \annee{1998--2000} DEUG sciences mention MIAS, \upmc.
  \annee{1998} Lauréat du concours général des lycées (sciences physiques).
\end{dateliste}

\partie{Activités d'enseignement}

\contexte{ATER à l'\upmc}
\begin{dateliste}
  \annee{2008--2009} LM204 « apprentissage et pratique de \LaTeX », deuxième
  année de licence.\\ Responsable de l'UE (nouvelle) : cours, TD, polycopié,
  télé-enseignement.
\end{dateliste}

\contexte{Formateur au CIES Jussieu}
\begin{dateliste}
  \annee{2007--2009} Formations \LaTeX{} niveaux intermédiaire et avancé,
  cours magistral.
  \annee{2006--2007} Proposition, définition et mise en place de formations
  \LaTeX.\\ Niveaux débutant et avancé, cours magistral.
\end{dateliste}

\contexte{Moniteur à l'\upmc}
\begin{dateliste}
  \annee{2007--2008} LM110 « fonctions », première année de licence, TD.
  \annee{2006--2007} LM220 « arithmétique », deuxième année de licence, TD.
  \annee{2005--2006} LM223 « formes quadratiques et géométrie », deuxième
  année de licence, TD.
\end{dateliste}

\pagebreak

\partie{Exposés}
\begin{dateliste}
  \annee{2008} Exposé « autour du théorème de Siegel » au séminaire
  étudiant de théorie des nombres de l'IMJ.
  \annee{2008} Exposé introductif « la géométrie gouverne l'arithmétique »
  au séminaire des thésards de l'IMJ.
  \annee{2007} Exposé « inégalités diophantiennes » au groupe
  de travail étudiant de théorie des nombres de Bordeaux.
  \annee{2006} Deux exposés « une preuve ``élémentaire'' de
  l'ex-conjecture de Mordell » au séminaire étudiant de théorie des
  nombres de l'IMJ.
\end{dateliste}

\partie{Rencontres scientifiques}

\begin{dateliste}
  \annee{2007} Colloque « progrès récents en approximation diophantienne »
  à Luminy.
  \annee{2006} Colloque « jeunes chercheurs en théorie des nombres » à
  Rennes.
\end{dateliste}

\partie{Responsabilités}

\contexte{Milieu universitaire}
\begin{dateliste}
  \annee{2007--2009}
  Membre du bureau des doctorants de l'institut de mathématiques de Jussieu,
  co-responsable des aspects informatiques.
\end{dateliste}

\contexte{Vie associative et logiciels libres}
\begin{dateliste}
  \annee{2008--} Membre de l'équipe \TeX\thinspace Live (la distribution \TeX{}
  standard sous Unix).
  \annee{2008--2009} Membre du CA (webmaster, relations universités) de la
  LUDI-IDF, la ligue universitaire d'improvisation théâtrale basée à Jussieu.
\end{dateliste}

\partie{Autres compétences et loisirs}
\begin{dateliste}
  \annee{Langues} Anglais courant, allemand scolaire.
  \annee{Info} Administration système sous Linux : serveur DNS, mail (SMTP,
  IMAPS), web, base de données, XMPP (Jabber), Subversion, fichiers (SFTP).\\
  Langages étudiés ou pratiqués : HTML+CSS, JS, PHP+MySQL, Pascal, C, Perl,
  shell Unix, Lua, et\dots \TeX !
  \annee{Musique} Pianiste amateur. Ancien étudiant de la section
  professionnelle (cursus théorique) du conservatoire de Genève.
  \annee{Impro} Membre d'une ligue d'improvisation théâtrale, la LUDI-IDF, et
  d'une troupe dérivée, les Ours dans ta baignoire.
  \annee{Sport} Roller : déplacement urbain et randonnée.
\end{dateliste}

\end{document}
