\documentclass[11pt, twoside]{article}

% Copyright 2006 Nicolas Billerey et Manuel P�gouri�-Gonnard
% ce source est sous licence GNU/FDL (http://www.gnu.org/licenses/fdl.txt)
% la licence vous autorise � utiliser et/ou distribuer des version modifi�es ou non
% � condition de conserver la m�me licence pour les versions modifi�es.

\usepackage[latin1]{inputenc}

\usepackage[french]{babel}
\usepackage[T1]{fontenc} \usepackage{lmodern}

\usepackage{amssymb, amsmath, amsthm} \usepackage[mathscr]{eucal}

\usepackage[a4paper]{geometry}

\newcommand{\pmm}[1]{\ \text{#1}}
\newcommand{\std}[1]{\mathbf{#1}} \newcommand{\N}{\std{N}} \newcommand{\Z}{\std{Z}}
\newcommand{\Q}{\std{Q}} \newcommand{\R}{\std{R}} \newcommand{\C}{\std{C}}
\newcommand{\I}[1][]{\std{I}_{#1}} \newcommand{\ind}[1]{\std{1}_{#1}}
\renewcommand{\ge}{\geqslant} \renewcommand{\le}{\leqslant}
\newcommand{\eps}{\varepsilon} \newcommand{\truc}{\,\cdot\,}

\title{{\large \textsc{Universit� Pierre et Marie Curie} 2006--2007}}
\author{LM220 Maths-Info groupe 2}
\date{Interrogation �crite \no1}

\theoremstyle{definition} \newtheorem{exo}{Exercice}
\newtheorem{qc}{Question de cours}

\begin{document}
\maketitle \thispagestyle{empty}

\begin{exo}
On consid�re l'�quation
\begin{equation}\label{eq:1}
77a-26b=3,\quad\textrm{avec }(a,b)\in\Z^2.\tag{$\star$}
\end{equation}
\begin{enumerate}
\item Donner un couple $(u,v)\in\Z^2$ v�rifiant
\begin{equation*}
77u-26v=1.
\end{equation*}
\item R�soudre l'�quation (\ref{eq:1}).
\item En utilisant la premi�re question, montrer que $\overline{26}$ est
inversible dans l'anneau $\Z/77\Z$ et d�terminer son inverse.
\end{enumerate}
\end{exo}

\bigskip

\begin{exo}
On consid�re la suite $(F_n)_{n\ge 0}$ d�finie par
\begin{equation*}
F_0=F_1=1\quad\textrm{et}\quad\forall n\ge 1,\quad F_{n+1}=F_n+F_{n-1}.
\end{equation*}
\begin{enumerate}
\item Calculer $F_0,\dots,F_{10}$.
\item Montrer que si $a$ et $b$ sont deux entiers $\ge 1$ premiers entre eux, alors $a$ et $a+b$ le
sont aussi.
\item En d�duire par r�currence que pour tout entier $n\ge 0$, $F_n$ et
$F_{n+1}$ sont premiers entre eux.
\end{enumerate}
\end{exo}

\bigskip

\begin{qc}
Qu'appelle-t'on morphisme de groupes? D�finir la notion de sous-groupe.

\end{qc}

\end{document}
