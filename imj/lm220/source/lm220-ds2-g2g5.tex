\documentclass[11pt, twoside]{article}

% Copyright 2006 Nicolas Billerey et Manuel P�gouri�-Gonnard
% ce source est sous licence GNU/FDL (http://www.gnu.org/licenses/fdl.txt)
% la licence vous autorise � utiliser et/ou distribuer des version modifi�es ou non
% � condition de conserver la m�me licence pour les versions modifi�es.

\usepackage[latin1]{inputenc}

\usepackage[french]{babel}
\usepackage[T1]{fontenc} \usepackage{lmodern}

\usepackage{amssymb, amsmath, amsthm} \usepackage[mathscr]{eucal}

\usepackage[a4paper]{geometry}

\newcommand{\pmm}[1]{\ \text{#1}}
\newcommand{\std}[1]{\mathbf{#1}} \newcommand{\N}{\std{N}} \newcommand{\Z}{\std{Z}}
\newcommand{\Q}{\std{Q}} \newcommand{\R}{\std{R}} \newcommand{\C}{\std{C}}
\newcommand{\I}[1][]{\std{I}_{#1}} \newcommand{\ind}[1]{\std{1}_{#1}}
\renewcommand{\ge}{\geqslant} \renewcommand{\le}{\leqslant}
\newcommand{\eps}{\varepsilon} \newcommand{\truc}{\,\cdot\,}
\newcommand{\modulo}{\ensuremath{\text{ mod }}}

\theoremstyle{definition} \newtheorem{exo}{Exercice}
\newtheorem{qc}{Question de cours}
\renewcommand{\theqc}{\hspace{-.3em}}

\title{{\large \textsc{Universit� Pierre et Marie Curie} 2006--2007}}
\author{LM220 Maths-Info groupes 2 et 5}
\date{Interrogation \no2}

\begin{document}
\maketitle \thispagestyle{empty}

\begin{exo} ~\\
 \begin{enumerate}
  \item Expliquer pourquoi $11$ est inversible dans $\Z/24\Z$ et calculer son inverse.
  \item En d�duire que l'�quation $11x = 3$ admet $9$ pour \emph{unique} solution dans $\Z/24\Z$.
  \item R�soudre dans $\Z$ le syst�me de congruences suivant.
  \[
   \left\{ \begin{aligned} 11x & \equiv & 3 \modulo 24 \\ x & \equiv & 2 \modulo 29 \end{aligned} \right.
  \]
 \end{enumerate}
\end{exo}

\bigskip

\begin{exo} ~\\
 \begin{enumerate}
  \item Donner la liste compl�te des �l�ments du groupe $(\Z/14\Z)^*$.
  \item Montrer que $(\Z/14\Z)^*$ est cyclique d'ordre $6$.
  \item Combien a-t-il de g�n�rateurs ?
  \item Calculer le reste de la division euclidienne de $297^{365}$ par $14$.
 \end{enumerate}
\end{exo}

\bigskip

\begin{qc}
 �noncer le th�or�me chinois.
\end{qc}

\end{document}
