\documentclass[11pt, twoside]{article}

% Copyright 2006 Nicolas Billerey et Manuel P�gouri�-Gonnard
% ce source est sous licence GNU/FDL (http://www.gnu.org/licenses/fdl.txt)
% la licence vous autorise � utiliser et/ou distribuer des version modifi�es ou non
% � condition de conserver la m�me licence pour les versions modifi�es.

\usepackage[latin1]{inputenc}

\usepackage[french]{babel}
\usepackage[T1]{fontenc} \usepackage{lmodern}

\usepackage{amssymb, amsmath, amsthm} \usepackage[mathscr]{eucal}

\usepackage[a4paper]{geometry}


\newcommand{\pmm}[1]{\ \text{#1}}
\newcommand{\std}[1]{\mathbf{#1}} \newcommand{\N}{\std{N}} \newcommand{\Z}{\std{Z}}
\newcommand{\Q}{\std{Q}} \newcommand{\R}{\std{R}} \newcommand{\C}{\std{C}}
\newcommand{\I}[1][]{\std{I}_{#1}} \newcommand{\ind}[1]{\std{1}_{#1}}
\renewcommand{\ge}{\geqslant} \renewcommand{\le}{\leqslant}
\newcommand{\eps}{\varepsilon} \newcommand{\truc}{\,\cdot\,}
\newcommand{\modulo}{\ensuremath{\text{ mod }}}

\theoremstyle{definition} \newtheorem{exo}{Exercice}
\newtheorem{qc}{Question de cours}
\renewcommand{\theqc}{\hspace{-.3em}}
\newtheorem{prop}{Propositon}
\renewcommand{\theprop}{\hspace{-.3em}}
\setlength{\arraycolsep}{2pt}

\title{{\large \textsc{Universit� Pierre et Marie Curie} 2006--2007}}
\author{LM220 Maths-Info groupe 1}
\date{Interrogation �crite \no3}

\begin{document}
\maketitle
\thispagestyle{empty}

\begin{exo} ~\\
 \begin{enumerate}
 \item Soit $n=(a_k\ldots a_1a_0)_{10}$ un entier �crit en base $10$. Montrer que $n$ est divisible par $11$ si et seulement si $a_0-a_1+a_2+\cdots+(-1)^ka_k$ est �galement divisible par $11$.
\item Dans un cryptosyst�me utilisant la m�thode RSA avec la cl� publique $(n,e)=(319,187)$, on souhaite envoyer le message $M_1=12$. D�terminer le message cod� $C_1$ � envoyer.
\item D�terminer la cl� secr�te de ce cryptosyst�me et d�coder le message $C_2=18$.
\end{enumerate}
\end{exo}

\bigskip

\begin{exo}
On consid�re les deux polyn�mes suivants
\begin{equation*}
 P(X)=X^4+1\quad\textrm{et}\quad Q(X)=X^3+X^2+X+1.
\end{equation*}
\begin{enumerate}
 \item \'Ecrire une relation de Bezout entre $P$ et $Q$ dans $\Q\left[X\right]$. En d�duire le pgcd de $P$ et $Q$ dans $\Q\left[X\right]$.
\item Quel est le pgcd de $P$ et $Q$ dans $\left(\Z/2\Z\right)\left[X\right]$?
\end{enumerate}


\end{exo}

\bigskip

\begin{qc} ~\\
 D�montrer la proposition suivante.
\begin{prop}
 Soient $p$ et $q$ deux nombres premiers distincts. Posons $n=pq$. Soit $t$ un entier naturel congru � $1$ modulo $\varphi(n)$. Alors, on a
\begin{equation*}
 a^t\equiv a\pmod{n}\quad\textrm{quel que soit }a\in\Z.
\end{equation*}
\end{prop}

\end{qc}

\end{document}
