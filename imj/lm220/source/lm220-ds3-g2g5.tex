\documentclass[11pt, twoside]{article}

% Copyright 2006 Nicolas Billerey et Manuel P�gouri�-Gonnard
% ce source est sous licence GNU/FDL (http://www.gnu.org/licenses/fdl.txt)
% la licence vous autorise � utiliser et/ou distribuer des version modifi�es ou non
% � condition de conserver la m�me licence pour les versions modifi�es.

\usepackage[latin1]{inputenc}

\usepackage[french]{babel}
\usepackage[T1]{fontenc} \usepackage{lmodern}

\usepackage{amssymb, amsmath, amsthm} \usepackage[mathscr]{eucal}

\usepackage[a4paper]{geometry}

\newcommand*{\pmm}[1]{\ \text{#1}}
\newcommand*{\std}[1]{\mathbf{#1}} \newcommand{\N}{\std{N}} \newcommand{\Z}{\std{Z}}
\newcommand{\Q}{\std{Q}} \newcommand{\R}{\std{R}} \newcommand{\C}{\std{C}}
\newcommand{\I}[1][]{\std{I}_{#1}} \newcommand{\ind}[1]{\std{1}_{#1}}
\renewcommand{\ge}{\geqslant} \renewcommand{\le}{\leqslant}
\newcommand{\eps}{\varepsilon} \newcommand{\truc}{\,\cdot\,}
\newcommand*{\modulo}{\ensuremath{\text{ mod }}}

\theoremstyle{definition} \newtheorem{exo}{Exercice}
\newtheorem{qc}{Question de cours}
\renewcommand{\theqc}{\hspace{-.3em}}
\newtheorem{thm}{Th�or�me}
\renewcommand{\thethm}{\hspace{-.3em}}

\title{{\large \textsc{Universit� Pierre et Marie Curie} 2006--2007}}
\author{LM220 Maths-Info groupes 2 et 5}
\date{Interrogation �crite \no3}

\begin{document}
\maketitle
\thispagestyle{empty}

\begin{exo} ~\par
 \begin{enumerate}
 \item Calculer $76^{29}$ modulo $7$ et $76^{29}$ modulo $13$.
\item Dans un cryptosyst�me utilisant la m�thode RSA avec la cl� publique $(n,e)=(91,5)$, on souhaite envoyer le message $M_1=9$. D�terminer le message cod� $C_1$ � envoyer.
\item D�terminer la cl� secr�te de ce cryptosyst�me et d�coder le message $C_2=76$.
\end{enumerate}
\end{exo}

\bigskip

\begin{exo}
On consid�re les deux polyn�mes suivants de $\Q\left[X\right]$~:
\begin{equation*}
 P(X)=X^4+X^3+2X^2-X+3\quad\textrm{et}\quad Q(X)=X^3+1.
\end{equation*}
\begin{enumerate}
\item Calculer le pgcd de $P$ et $Q$.
\item En d�duire la factorisation de $P$ en produit de facteurs irr�ductibles dans $\Q\left[X\right]$.
\end{enumerate}


\end{exo}

\bigskip

\begin{qc} ~\\
 D�montrer le th�or�me suivant.
\begin{thm}
 Soient $K$ un corps et $I$ un id�al non nul de $K\left[X\right]$. Il existe un unique polyn�me unitaire ${P\in K\left[X\right]}$ tel que l'on ait $I=(P)$.
\end{thm}
\end{qc}

\end{document}
