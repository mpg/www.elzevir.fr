\documentclass[11pt, twoside]{article}

% Copyright 2006 Nicolas Billerey et Manuel P�gouri�-Gonnard
% ce source est sous licence GNU/FDL (http://www.gnu.org/licenses/fdl.txt)
% la licence vous autorise � utiliser et/ou distribuer des version modifi�es ou non
% � condition de conserver la m�me licence pour les versions modifi�es.

\usepackage[latin1]{inputenc}

\usepackage[french]{babel}
\usepackage[T1]{fontenc} \usepackage{lmodern}

\usepackage{amssymb, amsmath, amsthm} \usepackage[mathscr]{eucal}

\usepackage[a4paper]{geometry}


\newcommand*{\pmm}[1]{\ \text{#1}}
\newcommand*{\std}[1]{\mathbf{#1}} \newcommand{\N}{\std{N}} \newcommand{\Z}{\std{Z}}
\newcommand{\Q}{\std{Q}} \newcommand{\R}{\std{R}} \newcommand{\C}{\std{C}}
\newcommand\F{\std{F}}
\newcommand*{\I}[1][]{\std{I}_{#1}} \newcommand*{\ind}[1]{\std{1}_{#1}}
\renewcommand{\ge}{\geqslant} \renewcommand{\le}{\leqslant}
\newcommand{\eps}{\varepsilon} \newcommand{\truc}{\,\cdot\,}

\theoremstyle{definition} \newtheorem{exo}{Exercice}
\newtheorem{qc}{Question de cours}
\renewcommand{\theqc}{\hspace{-.3em}}

\title{{\large \textsc{Universit� Pierre et Marie Curie} 2006--2007}}
\author{LM220 Maths-Info groupes 2 et 5}
\date{Interrogation �crite \no4}

\begin{document}
\maketitle
\thispagestyle{empty}

\begin{exo}
 On consid�re l'anneau 
 \[
  A = \F_3[X] / (X^2 - X - 1) \pmm.
 \]
 Justifier que l'anneau $A$ est un corps, et qu'il poss�de $9$ �l�ments. �num�rer les �l�ments de $A$ et dresser sa table de multiplication.
\end{exo}

\bigskip

\begin{exo}
 On consid�re l'anneau
 \[
  K = \F_5[X] / (X^2 + 2) \pmm.
 \]
 \begin{enumerate}
  \item Montrer que $K$ est un corps. Donner sa caract�ristique ainsi que son cardinal.
  \item On note $\alpha$ la classe de $X$ dans $K$.
  \begin{enumerate}
   \item Quel est l'inverse de $2 + 3\alpha$ dans $K^\times$?
   \item Quels sont les ordres respectifs de $\alpha$ et $2 + \alpha$ dans $K^\times$?
   \item En d�duire un g�n�rateur de $K^\times$.
  \end{enumerate}
 \end{enumerate}
\end{exo}

\end{document}
