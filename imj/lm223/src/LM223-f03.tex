\documentclass[11pt, twoside]{article}

% Copyright 2006 Nicolas Billerey et Manuel P�gouri�-Gonnard
% ce source est sous licence GNU/FDL (http://www.gnu.org/licenses/fdl.txt)
% la licence vous autorise � utiliser et/ou distribuer des version modifi�es ou non
% � condition de conserver la m�me licence pour les versions ult�rieures.

\usepackage[latin1]{inputenc}

\usepackage[a4paper]{geometry}
\usepackage[french]{babel}
\usepackage[T1]{fontenc} \usepackage{lmodern}
\usepackage{amssymb, amsmath, amsthm} \usepackage[mathscr]{eucal}

\newcounter{feuille}

\usepackage{fancyhdr} \pagestyle{fancy}
\fancyhead{} \fancyfoot{} \setlength{\headheight}{14pt}
\fancyhead[RO, LE]{Feuille \thefeuille} \fancyhead[C]{UPMC
2005--2006} \fancyhead[LO, RE]{LM223} \fancyfoot[C]{\thepage}
\renewcommand{\headrulewidth}{0.1pt}
\title{{\large \textsc{Universit� Pierre et Marie Curie} 2005--2006}}
\author{LM223 groupes 1, 2, 5 et 6}
\date{Feuille \thefeuille}

\theoremstyle{definition} \newtheorem{exo}{Exercice}

\newcommand{\pmm}[1]{\ \text{#1}}
\newcommand{\std}[1]{\mathbf{#1}} \newcommand{\N}{\std{N}} \newcommand{\Z}{\std{Z}}
\newcommand{\Q}{\std{Q}} \newcommand{\R}{\std{R}} \newcommand{\C}{\std{C}}
\newcommand{\I}[1][]{\std{I}_{#1}} \newcommand{\ind}[1]{\std{1}_{#1}}
\renewcommand{\ge}{\geqslant} \renewcommand{\le}{\leqslant}
\newcommand{\eps}{\varepsilon} \newcommand{\truc}{\,\cdot\,}

\DeclareMathOperator{\im}{im} \DeclareMathOperator{\tr}{tr}

\begin{document}
\setcounter{feuille}{3} \maketitle

Dans toute la feuille, si $E$ est un $\std{K}$-espace vectoriel
($\std{K}=\R$ ou $\C$) et $n$ un entier $\ge 2$, on d�signe par
$\mathcal{L}_n(E)$ l'espace des formes $n$-lin�aires sur $E$,
c'est-�-dire l'espace des formes multilin�aires de $E^n$ dans
$\std{K}$.

\section{Formes multilin�aires}

\begin{exo}
Soient $E$ un $\R$-espace vectoriel, $n$ un entier $\ge 2$ et
$f\in\mathcal{L}_n(E)$.
\begin{enumerate}
\item Montrer que $f$ est altern�e si, et seulement si, $f$ est
antisym�trique.
\item Montrer que si $f$ est altern�e, pour tout $\lambda\in\R$ et pour tout couple $(i,j)$
d'�l�ments de $\{ 1,\dots,n\}$ avec $i<j$, on a
\begin{equation*}
f(x_1,\dots,x_n)=f(x_1,\dots,x_i+\lambda x_j,\dots,x_j,\dots,x_n).
\end{equation*}
\end{enumerate}
\end{exo}


\begin{exo}
Soit $E$ un $\std{K}$-espace vectoriel ($\std{K}=\R$ ou $\C$). On
note $\mathcal{S}_2(E)$ (resp. $\mathcal{A}_2(E)$) l'espace des
formes bilin�aires sym�triques (resp. antisym�triques) de $E$.
Montrer que~:
\begin{equation*}
\mathcal{L}_2(E)=\mathcal{S}_2(E)\oplus\mathcal{A}_2(E).
\end{equation*}
\end{exo}

\begin{exo}\label{exo:ps}
Dans chacun des exemples suivants, montrer que $\varphi$ est un
produit scalaire sur $E$.
\begin{enumerate}
  \item $E=\R^n$, $n\ge2$ et
  $\varphi(x,y)=\sum\limits_{i=1}^{n}x_iy_i$,
  \item $E=M_n(\R)$, $n\ge2$ et $\varphi(A,B)=\tr(^{t}AB)$,
  \item\label{ps:poly} $E=\R_n\lbrack X\rbrack$, $n\ge2$ et
  $\varphi(P,Q)=\int_0^1P(t)Q(t)dt$,
  \item $E=\ell^2(\N)=\big\{u=(u_n)_n\in\R\, \big\lvert\, \sum\limits_{n\ge
  0}u_n^2<+\infty\big\}$ et
  $\varphi(u,v)=\sum\limits_{n=0}^{\infty}u_nv_n$. \\
  \lbrack Dans le 4., commencer par montrer que $\varphi$ est bien
  d�finie.\rbrack
\end{enumerate}
\end{exo}

\begin{exo}
Soient $A$ et $B$ deux matrices de $M_n(\R)$, on pose
\begin{equation*}
    \varphi(A,B)=\tr(AB).
\end{equation*}
\begin{enumerate}
  \item La forme $\varphi$ est-elle bilin�aire? sym�trique?
  \item Si $A=((a_{ij}))_{1\le i,j\le n}$ et $B=((b_{ij}))_{1\le i,j\le
  n}$, montrer que~:
  \begin{equation*}
    \tr(AB)=\sum_{1\le i,j\le n}a_{ij}b_{ji}.
  \end{equation*}

  \item Supposons � pr�sent, $A$ sym�trique et $B$ antisym�trique.
  Montrer alors~:
  \begin{itemize}
    \item $\varphi(A,A)\ge 0$,
    \item $\varphi(B,B)\le 0$,
    \item $\varphi(A,B)=0$.
  \end{itemize}
  \item La forme $\varphi$ est-elle un produit scalaire?
\end{enumerate}
\end{exo}

\begin{exo}
On consid�re l'application suivante d�finie sur $\R^3\times\R^3$~:
\begin{equation*}
    \big((x,y,z),(x',y',z')\big)\mapsto \Bigg(\begin{vmatrix}y & z \\
    y' & z'\\ \end{vmatrix},\begin{vmatrix}z & x \\
    z' & x'\\ \end{vmatrix},\begin{vmatrix}x & y \\
    x' & y'\\ \end{vmatrix}\Bigg)=(x,y,z)\wedge(x',y',z').
\end{equation*}
\begin{enumerate}
  \item Montrer qu'elle est bilin�aire altern�e.
  \item Montrer que si $\overrightarrow{e_1}$ et $\overrightarrow{e_2}$ sont
  deux vecteurs de $\R^3$, alors~:
  \begin{eqnarray*}
    (\overrightarrow{e_1}\wedge\overrightarrow{e_2})\cdot\overrightarrow{e_1} &=& 0, \\
    (\overrightarrow{e_1}\wedge\overrightarrow{e_2})\cdot\overrightarrow{e_2} &=& 0.
  \end{eqnarray*}
  \item En d�duire que, si $e_1$ et $e_2$ sont lin�airement ind�pendants, \[\textrm{Vect}(\overrightarrow{e_1},\overrightarrow{e_2})=
  \textrm{Vect}(\overrightarrow{e_1}\wedge\overrightarrow{e_2})^\bot=\big\{\overrightarrow{x}\in\R^3\, \big\lvert\,
  (\overrightarrow{e_1}\wedge\overrightarrow{e_2})\cdot\overrightarrow{x}=0\big\}\].
\end{enumerate}
\noindent\emph{Application.} D�terminer une �quation du plan
engendr� par les vecteurs $\overrightarrow{e_1}=(1,2,-3)$ et
$\overrightarrow{e_2}=(-2,0,1)$.
\end{exo}

\begin{exo}
On reprend le produit scalaire \ref{ps:poly}. de l'exercice
\ref{exo:ps} avec $n=2$~:
\begin{equation*}
    \forall\, P,\, Q\in\R_2\lbrack X\rbrack,\quad
    \varphi(P,Q)=\int_0^1P(t)Q(t)dt.
    \end{equation*}
\'Ecrire la matrice de $\varphi$ dans la base canonique de
$\R_2\lbrack X\rbrack$. Faire de m�me dans la base
$\mathcal{B}=\big(1,X,X^2-X+\frac{1}{6}\big)$.
\end{exo}





\end{document}
