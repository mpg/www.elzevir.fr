\documentclass[11pt, twoside]{article}

% Copyright 2006 Nicolas Billerey et Manuel P�gouri�-Gonnard
% ce source est sous licence GNU/FDL (http://www.gnu.org/licenses/fdl.txt)
% la licence vous autorise � utiliser et/ou distribuer des version modifi�es ou non
% � condition de conserver la m�me licence pour les versions ult�rieures.

\usepackage[latin1]{inputenc}

\usepackage[a4paper]{geometry}
\usepackage[french]{babel}
\usepackage[T1]{fontenc} \usepackage{lmodern}
\usepackage{amssymb, amsmath, amsthm} \usepackage[mathscr]{eucal}


\newcounter{feuille}

\usepackage{fancyhdr} \pagestyle{fancy}
\fancyhead{} \fancyfoot{} \setlength{\headheight}{14pt}
\fancyhead[RO, LE]{Feuille \thefeuille} \fancyhead[C]{UPMC
2005--2006} \fancyhead[LO, RE]{LM223} \fancyfoot[C]{\thepage}
\renewcommand{\headrulewidth}{0.1pt}
\title{{\large \textsc{Universit� Pierre et Marie Curie} 2005--2006}}
\author{LM223 maths-info groupes 1, 2, 5 et 6 \\ LM223 maths groupes 1 et 2}
\date{Feuille \thefeuille}

\theoremstyle{definition} \newtheorem{exo}{Exercice}


\newcommand{\pmm}[1]{\ \text{#1}}
\newcommand{\std}[1]{\mathbf{#1}} \newcommand{\N}{\std{N}} \newcommand{\Z}{\std{Z}}
\newcommand{\Q}{\std{Q}} \newcommand{\R}{\std{R}} \newcommand{\C}{\std{C}}
\newcommand{\I}[1][]{\std{I}_{#1}} \newcommand{\ind}[1]{\std{1}_{#1}}
\renewcommand{\ge}{\geqslant} \renewcommand{\le}{\leqslant}
\newcommand{\eps}{\varepsilon} \newcommand{\truc}{\,\cdot\,}



\DeclareMathOperator{\im}{im} \DeclareMathOperator{\tr}{tr}

\begin{document}
\setcounter{feuille}{4} \maketitle

\begin{exo} 

Trouver les formes polaires et le rang des formes quadratiques suivantes sur $\mathbb{R}^4$
\begin{enumerate}
\item $q(x,y,z,t)=xy+y^2$ 
\item $q(x,y,z,t)=xy+zt+t^2$ 
\item $q(x,y,z,t)=x^2-y^2+z^2-t^2$
\end{enumerate}
\end{exo}

\vspace{0.2cm}
\begin{exo}
Soit $f(x,y)=x_1y_1-\frac{3}{2}x_1y_2-\frac{3}{2}x_2y_1+6x_2y_2$ sur $\mathbb{R}^2$\\
Montrer que f est bilin�aire sym�trique.
\end{exo}


\vspace{0.2cm}
\begin{exo}
Soit q la forme quadratique sur un $\mathbb{K}$-ev associ�e � la forme bilin�aire sym�trique f.
\begin{enumerate}
  \item V�rifier que $f(u,v)=\frac{1}{2}\left(q(u+v)-q(u)-q(v)\right)$ (forme polaire de f)
  \item Montrer que $\forall(x,y,z)\in E^3$, on a $$q(x+y)+q(y+z)+q(z+x)=q(x)+q(y)+q(z)+q(x+y+z)$$
  \end{enumerate}
\end{exo}

\vspace{0.2cm}
\begin{exo}
Soit $q(x)=2x_1^2+6x_1x_2-2x_1x_3+4x_2x_3+x_2^2-3x_3^2$ sur $\mathbb{R}^3$
D�terminer son rang et son noyau. Est-elle non d�g�n�r�e?
\end{exo}

\vspace{0.2cm}
\begin{exo}
Soient A et B sym�triques telles que $\forall$ X $\in \mathbb{R}^4$, $^tXAX=$ $^tXBX$ (*)
\begin{enumerate}
\item Soient X, Y $\in \mathbb{R}^4$, en appliquant (*) � X, Y et X+Y, montrer que $^tXAY=$ $^tXBY$
\item En rempla�ant X et Y par des vecteurs de la base canonique, en d�duire que A=B
  \end{enumerate}
\end{exo}

\vspace{0.2cm}
\begin{exo}
D�composer en sommes de carr�s de formes lin�aires ind�pendantes les formes quadratiques sur $\mathbb{R}^3$ suivantes
\begin{enumerate}
\item $q(x,y,z)=x^2+y^2+xz$
\item $q(x,y,z)=xy+3xz$
  \end{enumerate}
\end{exo}

\end{document}
