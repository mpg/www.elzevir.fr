\documentclass[11pt, twoside]{article}

% Copyright 2006 Nicolas Billerey et Manuel P�gouri�-Gonnard
% ce source est sous licence GNU/FDL (http://www.gnu.org/licenses/fdl.txt)
% la licence vous autorise � utiliser et/ou distribuer des version modifi�es ou non
% � condition de conserver la m�me licence pour les versions ult�rieures.

\usepackage[latin1]{inputenc}

\usepackage[a4paper]{geometry}
\usepackage[french]{babel}
\usepackage[T1]{fontenc} \usepackage{lmodern}
\usepackage{amssymb, amsmath, amsthm} \usepackage[mathscr]{eucal}


\newcounter{feuille}

\usepackage{fancyhdr} \pagestyle{fancy}
\fancyhead{} \fancyfoot{} \setlength{\headheight}{14pt}
\fancyhead[RO, LE]{Feuille \thefeuille} \fancyhead[C]{UPMC
2005--2006} \fancyhead[LO, RE]{LM223} \fancyfoot[C]{\thepage}
\renewcommand{\headrulewidth}{0.1pt}
\title{{\large \textsc{Universit� Pierre et Marie Curie} 2005--2006}}
\author{LM223 maths-info groupes 1, 2, 5 et 6 \\ LM223 maths groupes 1 et 2}
\date{Feuille \thefeuille}

\theoremstyle{definition} \newtheorem{exo}{Exercice}


\newcommand{\pmm}[1]{\ \text{#1}}
\newcommand{\std}[1]{\mathbf{#1}} \newcommand{\N}{\std{N}} \newcommand{\Z}{\std{Z}}
\newcommand{\Q}{\std{Q}} \newcommand{\R}{\std{R}} \newcommand{\C}{\std{C}}
\newcommand{\I}[1][]{\std{I}_{#1}} \newcommand{\ind}[1]{\std{1}_{#1}}
\renewcommand{\ge}{\geqslant} \renewcommand{\le}{\leqslant}
\newcommand{\eps}{\varepsilon} \newcommand{\truc}{\,\cdot\,}

\DeclareMathOperator{\im}{im} \DeclareMathOperator{\tr}{tr}


\begin{document}
\setcounter{feuille}{5} \maketitle

\begin{exo} 
Calculer le rang et la signature de chacune des  formes quadratiques de l'exercice 6 feuille d'exercices 4.
\end{exo}



\vspace{0.2cm}
\begin{exo}
~~ \newline
\begin{enumerate}
\item Trouver un changement de coordonn�es qui diagonalise la forme quadratique
\[ q: \mathbb{R}^4 \rightarrow \mathbb{R} , \  
        q(\begin{pmatrix} x_1\\ x_2\\ x_3\\ x_4 \end{pmatrix})
        =x_1^2+2x_1x_2-4x_1x_4+3x_2^2+8x_2x_3+6x_3^2-2x_4^2 \ .\]
\item D�terminer la forme polaire de $q$ (=la forme bilin�aire sym�trique associ�e � $q$).
\item D�terminer le rang de $q$.
\item D�terminer la signature de $q$.
\item D�terminer le noyau de $q$.
\end{enumerate}
\end{exo}


\vspace{0.2cm}
\begin{exo}
~~ \newline
\begin{enumerate}
  \item En utilisant une d�composition en carr�s, diagonaliser la forme quadratique
  \[q: \mathbb{R}^4 \rightarrow \mathbb{R} ,\quad q(X)=4xy+4yz-2zt ,\quad X=
        \begin{pmatrix}x\\ y\\ z\\ t \end{pmatrix} \in \mathbb{R}^4\,.\]
  \item D�terminer la matrice de changement de base (ou son inverse) qui correspond � cette diagonalisation.
  \item D�terminer la signature de $q$.
  \end{enumerate}
\end{exo}

\vspace{0.2cm}
\begin{exo}
On d�signe par $\alpha$ et $\beta$ deux r�els tels que $\alpha^2+\beta^2=1$. Discuter suivant les valeurs de $\alpha$ et $\beta$ le rang et la signature de la forme quadratique sur $\mathbb{R}^3$ dont la matrice dans la base canonique est 
\[ \begin{pmatrix}
   {1} & {\alpha} & {0} \cr
   {\alpha} & {1} & {\beta} \cr
   {0} & {\beta} & {\alpha+\beta} 
   \end{pmatrix}       
\]       
(On utilisera le proc�d� d'orthogonalisation de Gauss et on distinguera les cas o� $\beta=0$). Repr�senter graphiquement sur le cercle $\alpha^2+\beta^2=1$ les diff�rents cas.
\end{exo}

\vspace{0.2cm}
\begin{exo}
Soient $E=M_n(\mathbb{R}),\ q(X)=Tr(X^2)\quad (X \in E)$.
\begin{enumerate}
\item D�terminer la forme polaire $f:E \times E \rightarrow \mathbb{R}$ de $q$.
\item Montrer que si A est sym�trique non nulle (resp. antisym�trique), alors $q(A)>0$ (resp. $q(A)<0$).
\item Montrer que si A est sym�trique, B antisym�trique, alors $f(A,B)=0$.
\item En d�duire le rang et la signature de $q$.

  [quelle est la dimension de l'espace des matrices sym�triques? antisym�triques?]
\end{enumerate}
\end{exo}

\vspace{0.2cm}
\begin{exo}
~~ \newline
\begin{enumerate}
\item Trouver un exemple de trois sous-espaces vectoriels $F_1$, $F_2$, $F_3$ $\in E$ tels que $F_1 \cap F_2 = F_2 \cap F_3=F_1 \cap F_3=\{0\}$ alors que la somme $F_1+F_2+F_3$ n'est pas directe.
\item Soit $\Phi$ une forme quadratique d�finie, on suppose que $F_1$, $F_2$, $F_3$ sont deux � deux $\Phi$-orthogonaux. Montrer que $F_1$, $F_2$ et $F_3$ sont en somme directe.
  \end{enumerate}
\end{exo}


\vspace{0.2cm}
\begin{exo}
~~ \newline
\begin{enumerate}
\item Soit $q: E \rightarrow K$ une forme quadratique sur un espace vectoriel de dimension finie. Pour tout sous-espace vectoriel $F \subset E$, montrer que l'on a ($F^\perp)^\perp=F+N(q)$.
\item Soit $q: E \rightarrow K$ une forme quadratique et $F$, $G \subset E$ des sous espaces vectoriels; montrer qu'alors $(F+G)^{\perp}=F^{\perp} \cap G^{\perp}$. Lorsque $\dim(E)<\infty$ et $q$ est non-d�g�n�r�e, montrer que l'on a $( F \cap G)^{\perp} =F^{\perp}+G^{\perp}$. 
\end{enumerate}
\end{exo}

\end{document}
