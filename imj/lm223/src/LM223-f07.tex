\documentclass[11pt, twoside]{article}

% Copyright 2006 Nicolas Billerey et Manuel P�gouri�-Gonnard
% ce source est sous licence GNU/FDL (http://www.gnu.org/licenses/fdl.txt)
% la licence vous autorise � utiliser et/ou distribuer des version modifi�es ou non
% � condition de conserver la m�me licence pour les versions ult�rieures.

\usepackage[latin1]{inputenc}

\usepackage[a4paper]{geometry}
\usepackage[french]{babel}
\usepackage[T1]{fontenc} \usepackage{lmodern}
\usepackage{amssymb, amsmath, amsthm} \usepackage[mathscr]{eucal}



\newcounter{feuille}

\usepackage{fancyhdr} \pagestyle{fancy}
\fancyhead{} \fancyfoot{} \setlength{\headheight}{14pt}
\fancyhead[RO, LE]{Feuille \thefeuille} \fancyhead[C]{UPMC
2005--2006} \fancyhead[LO, RE]{LM223} \fancyfoot[C]{\thepage}
\renewcommand{\headrulewidth}{0.1pt}
\title{{\large \textsc{Universit� Pierre et Marie Curie} 2005--2006}}
\author{LM223 maths-info groupes 1, 2, 5 et 6 \\ LM223 maths groupes 1 et 2}
\date{Feuille \thefeuille}

\theoremstyle{definition} \newtheorem{exo}{Exercice}


\newcommand{\pmm}[1]{\ \text{#1}}
\newcommand{\std}[1]{\mathbf{#1}} \newcommand{\N}{\std{N}} \newcommand{\Z}{\std{Z}}
\newcommand{\Q}{\std{Q}} \newcommand{\R}{\std{R}} \newcommand{\C}{\std{C}}
\newcommand{\I}[1][]{\std{I}_{#1}} \newcommand{\ind}[1]{\std{1}_{#1}}
\renewcommand{\ge}{\geqslant} \renewcommand{\le}{\leqslant}
\newcommand{\eps}{\varepsilon} \newcommand{\truc}{\,\cdot\,}



\DeclareMathOperator{\im}{im} \DeclareMathOperator{\tr}{tr}

\begin{document}
\setcounter{feuille}{7} \maketitle

\begin{exo} 
Faire l'�tude compl�te de la forme quadratique sur $\R^4$ d�finie par :
\[ q(x, y, z, t) = x^2 + 2xz + 2xt + yz + ty + z^2 + 3 zt + t^2 \pmm.\]
\end{exo}

\begin{exo}
Montrer que toute forme quadratique d�finie est non-d�g�n�r�e. La r�ciproque est-elle vraie?
\end{exo}

\begin{exo}
On consid�re la forme quadratique sur $\R^2$ donn�e par $q(x, y) = x^2 - y^2$.
\begin{enumerate}
 \item Donner son rang et sa signature. Est-elle d�g�n�r�e? Est-elle d�finie?
 \item Calculer et dessiner son c�ne isotrope.
 \item Pour $a \in \R$, on pose $D_a = \text{Vect}(\binom{1}{a})$ la droite de pente $a$ passant par l'origine, et $D_\infty = \text{Vect}(\binom{0}{1})$ la droite verticale passant par l'origine. Montrer que $D_a^\perp = D_{1/a}$, $D_0^\perp = D_\infty$ et r�ciproquement $D_\infty^\perp = D_0$.
 \item Trouver une valeur de $a$ telle que $D_a + D_a^\perp \neq \R^2$ et $D_a \cap D_a^\perp \neq \{0\}$. 
\end{enumerate}
\end{exo}

\begin{exo}
On consid�re $E = \mathcal{C}([-1; 1], \R)$ l'espace des fonctions continues sur $[-1; 1]$ � valeurs r�elles, muni de la forme quadratique \[q : f \mapsto \int_{-1}^1 t f^2(t) dt\pmm.\]
Montrer que les fonctions paires ou impaires sont des vecteurs isotropes de $q$.
\end{exo}

\end{document}
