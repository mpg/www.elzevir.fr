\documentclass[11pt, twoside]{article}

% Copyright 2006 Nicolas Billerey et Manuel P�gouri�-Gonnard
% ce source est sous licence GNU/FDL (http://www.gnu.org/licenses/fdl.txt)
% la licence vous autorise � utiliser et/ou distribuer des version modifi�es ou non
% � condition de conserver la m�me licence pour les versions ult�rieures.

\usepackage[latin1]{inputenc}

\usepackage[a4paper]{geometry}
\usepackage[french]{babel}
\usepackage[T1]{fontenc} \usepackage{lmodern}
\usepackage{amssymb, amsmath, amsthm} \usepackage[mathscr]{eucal}



\newcounter{feuille}

\usepackage{fancyhdr} \pagestyle{fancy}
\fancyhead{} \fancyfoot{} \setlength{\headheight}{14pt}
\fancyhead[RO, LE]{Feuille \thefeuille} \fancyhead[C]{UPMC
2005--2006} \fancyhead[LO, RE]{LM223} \fancyfoot[C]{\thepage}
\renewcommand{\headrulewidth}{0.1pt}
\title{{\large \textsc{Universit� Pierre et Marie Curie} 2005--2006}}
\author{LM223 maths-info groupes 1, 2, 5 et 6 \\ LM223 maths groupes 1 et 2}
\date{Feuille \thefeuille}

\theoremstyle{definition} \newtheorem{exo}{Exercice}


\newcommand{\pmm}[1]{\ \text{#1}}
\newcommand{\std}[1]{\mathbf{#1}} \newcommand{\N}{\std{N}} \newcommand{\Z}{\std{Z}}
\newcommand{\Q}{\std{Q}} \newcommand{\R}{\std{R}} \newcommand{\C}{\std{C}}
\newcommand{\I}[1][]{\std{I}_{#1}} \newcommand{\ind}[1]{\std{1}_{#1}}
\renewcommand{\ge}{\geqslant} \renewcommand{\le}{\leqslant}
\newcommand{\eps}{\varepsilon} \newcommand{\truc}{\,\cdot\,}



\DeclareMathOperator{\im}{im} \DeclareMathOperator{\tr}{tr}

\begin{document}
\setcounter{feuille}{8} \maketitle

\begin{exo} 
\begin{enumerate}
\item Lesquelles, parmi les matrices suivantes, sont orthogonales?
\begin{equation*}
A=\frac{1}{2}\begin{pmatrix}
\sqrt{3} & 1 \\
1 & -\sqrt{3} \\
\end{pmatrix},
\quad
B=\frac{1}{2}\begin{pmatrix}
\sqrt{3} & -1 \\
1 & -\sqrt{3} \\
\end{pmatrix},
\quad
C=\frac{1}{2}\begin{pmatrix}
\sqrt{3} & -1 \\
-1 & -\sqrt{3} \\
\end{pmatrix},
\quad
D=\frac{1}{2}\begin{pmatrix}
\sqrt{3} & -1 \\
-1 & \sqrt{3} \\
\end{pmatrix}.
\quad
\end{equation*}
\item Pour toute matrice orthogonale de 1., d�crire g�om�triquement l'isom�trie
correspondante de $\R^2$.
\item \'Ecrire les isom�tries de 2. sous leur forme complexe.
\end{enumerate}
\end{exo}

\begin{exo}
Pour $w=u+iv$ un nombre complexe, on note
\begin{equation*}
M(w)=\begin{pmatrix}
u & -v \\
v & u \\
\end{pmatrix}
\end{equation*}
la matrice repr�sentant l'application de multiplication par $w$.

D�terminer les vecteurs propres de la matrice $M(w)$. Est-elle diagonalisable?
\end{exo}

\begin{exo}
\begin{enumerate}
\item Montrer que les matrices suivantes sont orthogonales~:
\begin{eqnarray*}
A=\frac{1}{3}\begin{pmatrix}
1 & -2 & -2 \\
-2 & 1 & -2 \\
-2 & -2 & 1 \\
\end{pmatrix},
\quad
B=\frac{1}{3}\begin{pmatrix}
1 & -2 & 2 \\
-2 & 1 & 2 \\
-2 & -2 & -1 \\
\end{pmatrix},
\\
C=\frac{1}{3}\begin{pmatrix}
1 & 2 & 2 \\
-2 & -1 & 2 \\
-2 & 2 & -1 \\
\end{pmatrix},
\quad
D=\frac{1}{3}\begin{pmatrix}
-1 & 2 & 2 \\
2 & -1 & 2 \\
2 & 2 & -1 \\
\end{pmatrix}. \\
\end{eqnarray*}
\item Pour chacune des matrices de 1., d�crire g�om�triquement l'isom�trie de
$\R^3$ correspondante.
\end{enumerate}
\end{exo}

\begin{exo}
Soit $A\in O(3)$ une matrice orthogonale de $M_3(\R)$ de d�terminant $-1$.
Montrer que $-1$ est valeur propre de $A$.
\end{exo}

\begin{exo}
Dans l'espace euclidien $\R^3$ muni du produit scalaire usuel, on note $P$ le
plan d'�quation
\begin{equation*}
x+y+z=0.
\end{equation*}
\'Ecrire la matrice dans la base canonique de la sym�trie orthogonale par
rapport � $P$.
\end{exo}

\end{document}
