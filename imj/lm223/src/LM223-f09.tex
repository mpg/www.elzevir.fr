\documentclass[11pt, twoside]{article}

% Copyright 2006 Nicolas Billerey et Manuel P�gouri�-Gonnard
% ce source est sous licence GNU/FDL (http://www.gnu.org/licenses/fdl.txt)
% la licence vous autorise � utiliser et/ou distribuer des version modifi�es ou non
% � condition de conserver la m�me licence pour les versions ult�rieures.

\usepackage[latin1]{inputenc}

\usepackage[a4paper]{geometry}
\usepackage[french]{babel}
\usepackage[T1]{fontenc} \usepackage{lmodern}
\usepackage{amssymb, amsmath, amsthm} \usepackage[mathscr]{eucal}


\newcounter{feuille}

\usepackage{fancyhdr} \pagestyle{fancy}
\fancyhead{} \fancyfoot{} \setlength{\headheight}{14pt}
\fancyhead[RO, LE]{Feuille \thefeuille} \fancyhead[C]{UPMC
2005--2006} \fancyhead[LO, RE]{LM223} \fancyfoot[C]{\thepage}
\renewcommand{\headrulewidth}{0.1pt}
\title{{\large \textsc{Universit� Pierre et Marie Curie} 2005--2006}}
\author{LM223 maths-info groupes 1, 2, 5 et 6 \\ LM223 maths groupes 1 et 2}
\date{Feuille \thefeuille}

\theoremstyle{definition} \newtheorem{exo}{Exercice}


\newcommand{\pmm}[1]{\ \text{#1}}
\newcommand{\std}[1]{\mathbf{#1}} \newcommand{\N}{\std{N}} \newcommand{\Z}{\std{Z}}
\newcommand{\Q}{\std{Q}} \newcommand{\R}{\std{R}} \newcommand{\C}{\std{C}}
\newcommand{\I}[1][]{\std{I}_{#1}} \newcommand{\ind}[1]{\std{1}_{#1}}
\renewcommand{\ge}{\geqslant} \renewcommand{\le}{\leqslant}
\newcommand{\eps}{\varepsilon} \newcommand{\truc}{\,\cdot\,}

\DeclareMathOperator{\im}{im} \DeclareMathOperator{\tr}{tr}


\begin{document}
\setcounter{feuille}{9} \maketitle

\begin{exo}
Trouver une rotation $P\in O^+(3)$ qui diagonalise la forme quadratique suivante~:
\begin{equation*}
q~: \R^3\longrightarrow \R;\quad  X=\begin{pmatrix} x \\ y \\
z\end{pmatrix}\longmapsto
q(X)=x^2-16xy+8xz+y^2+8yz+7z^2. 
\end{equation*}
En d�duire la signature de $q$.
\end{exo}

\begin{exo}
Diagonaliser dans une base orthonorm�e \emph{pour le produit scalaire usuel} les
matrices suivantes~:
\begin{equation*}
A=\begin{pmatrix}2 & -4 & 2 \\ -4 & 2 & 2 \\ 2 & 2 & 5\end{pmatrix},\quad 
B=\begin{pmatrix}1 & 0 & -1 \\ 0 & 1 & \sqrt{2} \\ -1 & \sqrt{2} &
-1\end{pmatrix}. 
\end{equation*}
\end{exo}

\begin{exo}
On consid�re la forme quadratique suivante~:
\begin{equation*}
q~: \R^3\longrightarrow \R;\quad  X=\begin{pmatrix} x \\ y \\
z\end{pmatrix}\longmapsto
q(X)=x^2+5y^2+4xy-2yz+z^2. 
\end{equation*}
Dans une base orthonorm�e (\emph{pour le produit scalaire usuel}) bien choisie, d�crire g�om�triquement l'ensemble 
\begin{equation*}
\mathcal{E}=\lbrace (x,y,z)\in\R^3\ \arrowvert\ q(x,y,z)=1\rbrace.
\end{equation*} 
\end{exo}

\begin{exo}
Soit $u$ un endomorphisme d'un espace vectoriel euclidien $E$ et $F$ un
sous-espace de $E$. On suppose que $u(F)\subset F$. On note $^tu$ l'adjoint de
$u$. Montrer que l'on a alors~: 
\begin{equation*}
^tu(F^{\perp})\subset F^{\perp}.
\end{equation*}
\end{exo}

\begin{exo}
On consid�re dans $\R_2\lbrack X\rbrack$ les produits scalaires suivants~:
\begin{eqnarray*}
\langle P,Q\rangle & = & a_0b_0+a_1b_1+a_2b_2,\quad \textrm{si}\ P=a_0+a_1X+a_2X^2\ \textrm{et}\
Q=b_0+b_1X+b_2X^2, \\ 
(P \arrowvert Q) & = & \int_0^1P(t)Q(t)dt. 
\end{eqnarray*}
On note $D$ l'op�rateur de d�rivation. Calculer l'adjoint de $D$ \emph{pour le
produit scalaire} $\langle\truc ,\truc\rangle$. On d�finit l'endomorphisme $A$ par
$$\left\lbrace\begin{array}{ccl}
A(1) & = & 12X-6, \\
A(X) & = & 30X^2-24X+2, \\
A(X^2) & = & 30X^2-26X+3.
\end{array}\right.$$
Montrer que $A$ est l'adjoint de $D$ \emph{pour le produit scalaire} $(\truc
\arrowvert\truc)$.
\end{exo}

\end{document}
