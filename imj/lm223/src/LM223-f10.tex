\documentclass[11pt, twoside]{article}

% Copyright 2006 Nicolas Billerey et Manuel P�gouri�-Gonnard
% ce source est sous licence GNU/FDL (http://www.gnu.org/licenses/fdl.txt)
% la licence vous autorise � utiliser et/ou distribuer des version modifi�es ou non
% � condition de conserver la m�me licence pour les versions ult�rieures.

\usepackage[latin1]{inputenc}

\usepackage[a4paper]{geometry}
\usepackage[french]{babel}
\usepackage[T1]{fontenc} \usepackage{lmodern}
\usepackage{amssymb, amsmath, amsthm} \usepackage[mathscr]{eucal}


\newcounter{feuille}

\usepackage{fancyhdr} \pagestyle{fancy}
\fancyhead{} \fancyfoot{} \setlength{\headheight}{14pt}
\fancyhead[RO, LE]{Feuille \thefeuille} \fancyhead[C]{UPMC
2005--2006} \fancyhead[LO, RE]{LM223} \fancyfoot[C]{\thepage}
\renewcommand{\headrulewidth}{0.1pt}
\title{{\large \textsc{Universit� Pierre et Marie Curie} 2005--2006}}
\author{LM223 maths-info groupes 1, 2, 5 et 6 \\ LM223 maths groupes 1 et 2}
\date{Feuille \thefeuille}

\theoremstyle{definition} \newtheorem{exo}{Exercice}


\newcommand{\pmm}[1]{\ \text{#1}}
\newcommand{\std}[1]{\mathbf{#1}} \newcommand{\N}{\std{N}} \newcommand{\Z}{\std{Z}}
\newcommand{\Q}{\std{Q}} \newcommand{\R}{\std{R}} \newcommand{\C}{\std{C}}
\newcommand{\I}[1][]{\std{I}_{#1}} \newcommand{\ind}[1]{\std{1}_{#1}}
\renewcommand{\ge}{\geqslant} \renewcommand{\le}{\leqslant}
\newcommand{\eps}{\varepsilon} \newcommand{\truc}{\,\cdot\,}

\DeclareMathOperator{\im}{im} \DeclareMathOperator{\tr}{tr}


\begin{document}
\setcounter{feuille}{10} \maketitle

\begin{exo}
Dites si les applications suivantes sont des formes hermitiennes. Si c'est cas, donner leur forme quadratique hermitienne associ�e. 
\begin{enumerate}
 \item $A, B \in \std{M}_n(\C) \mapsto \tr(A\overline B)$,
 \item $X, Y \in \C^n \mapsto {}^t\!X\overline Y$,
 \item $(x_1, x_2), (y_1, y_2) \mapsto 2x_1\overline{y_1}+ i x_1\overline{y_2} - i x_2\overline{y_1} + 5 x_2 \overline{y_2}$,
 \item $(x_1, x_2), (y_1, y_2) \mapsto (1+i)x_1\overline{y_1}+ 2 x_1\overline{y_2} + 2 x_2\overline{y_1} + 5i x_2 \overline{y_2}$,
 \item $(x_1, x_2, x_3), (y_1, y_2, y_3) \mapsto x_1\overline{y_1}+ (1+i) x_1\overline{y_2} + (1-i) x_2\overline{y_1} + 5 x_1 \overline{y_3} + 5 x_3 \overline{y_1} + 2 x_2 \overline{y_2} + i x_2 \overline{y_3} - i x_3 \overline{y_2} + 7 x_3 \overline{y_3}$,
 \item $(x_1, x_2), (y_1, y_2) \mapsto x_1\overline{y_2} +  y_2\overline{x_1}$.
\end{enumerate}
\end{exo}

\begin{exo}
Montrer que les coefficients diagonaux d'une matrice hermitienne sont r�els.
\end{exo}

\begin{exo}
On consid�re la forme quadratique $q$ definie sur $\C^2$ par 
\[
q(x, y) = x^2 - y^2 \pmm.
\]
Trouver une base dans laquelle la matrice de $q$ est l'identit�.
\end{exo}

\begin{exo}
Faire l'�tude compl�te de la forme quadratique hermitienne $q$ sur $\C^3$ d�finie par :
\[
q(x, y, z) = x \bar x + y \bar y + z \bar z + x \bar y + y \bar x - y \bar z - z \bar y \pmm.
\]
\end{exo}

\begin{exo}
Montrer que $SU(2) = \left\{ \begin{pmatrix} a & -\bar c \\ c & \bar a \end{pmatrix} \quad a,\ c \in \C \text{ tels que } \lvert a \rvert^2 + \lvert b \rvert ^2 = 1 \right\}$. 
\end{exo}

\begin{exo}
Diagonaliser les matrices hermitiennes suivantes en base orthonorm�e pour le produit scalaire hermitien usuel :
\[
A = \begin{pmatrix} 4 & i & i \\ -i & 4 & 1 \\ -i & 1 & 4 \end{pmatrix} \quad 
B = \begin{pmatrix} 1 & 0 & i \\ 0 & 1 & -1 \\ -i & -1 & 1 \end{pmatrix} \pmm.
\]
\end{exo}

\begin{exo}
Montrer que la matrice suivante est unitaire et la diagonaliser en base orthonorm�e :
\[
M = \frac 1 3 \begin{pmatrix} 1+2i & 1+i & 1+i \\ -1-i & 1+2i & 1-i \\ -1-i & 1-i & 1+2i \end{pmatrix} \pmm.
\]
\end{exo}

\end{document}
